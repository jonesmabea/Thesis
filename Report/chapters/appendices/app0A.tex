%
% file: localoperator.tex
% author: Victor Brena
% description: Briefly describes properties of the local operator.
%

\chapter{Appendix A}


\begin{table}[H]
	\centering
		\resizebox{\textwidth}{!}{%
		\begin{tabular}{|l|l|l|l|l|l|l|}
			\hline
			\textbf{LiDAR}       & \textbf{Hor FOV} & \textbf{Ver FOV} & \textbf{Range} & \textbf{Angular Resolution} & \textbf{Points/second}          & \textbf{Channels} \\ \hline
			\textit{\textbf{VLS-128}} & 360\degree                    & +15\degree to -25\degree          & 300m           & 0.11\degree                               & $\sim$9.6 Million & 128               \\ \hline
			\textit{\textbf{HDL-64E}} & 360\degree                    & 26.9\degree                 & 120m           & $\sim$0.4\degree                          & $\sim$2.2 Million                   & 64                \\ \hline
			\textit{\textbf{HDL-32E}} & 360\degree                    & +10\degree to -30\degree          & 80m-100m       & 0.1\degree                                & $\sim$1.39 Million                  & 32                \\ \hline
			\textit{\textbf{VLP-32C}} & 360\degree                    & +15\degree to -25\degree          & 200m           & 0.1\degree                                & $\sim$1.2 Million                   & 32                \\ \hline
			\textit{\textbf{VLP-16}}  & 360\degree                    & $\pm$ 15\degree                 & 100m           & 0.1\degree                                & 600,000                             & 16                \\ \hline
	\end{tabular}
	}
	
	\caption{Velodyne LiDAR Family}
	\label{velodyne-table}
\end{table}


\section{Code Listings}
\begin{lstlisting}[language=bash, caption={.pcap -> .pcd},label={lst:ros}]
	#Code to read .pcap files and convert them to .pcd 
	
	#Load existing pcap file and create a sensor publisher
	$ roslaunch velodyne_pointcloud VLP-16-points.launch pcap:=(path to .pcap file) 
	
	#Receiver to  convert sensor messages to .pcd (In separate terminal window)
	$ rosrun pcl_ros pointcloud_to_pcd input:=/velodyne_points _prefix:=(path to save .pcd file) 
	
\end{lstlisting}


\begin{lstlisting}[language=Python, caption=Calculating FLOPS and Parameters,label={lst:profiler}]
def calculate_metrics(chkpoint):
	"""
	Profiles a Tensorflow Graph to obtain the parameters and FLOPS
	params:
	chkpoint: Tensorflow checkpoint. 
	returns: 
	stats: FLOPS proto object 
	params: Parameters proto objects
	
"""
	with tf.Graph().as_default():
		#GPU options. 
		gpu_options = tf.GPUOptions(per_process_gpu_memory_fraction=0.4,
																visible_device_list='0',
																allow_growth=True)
		#create session config proto object
		config = tf.ConfigProto(
							gpu_options=gpu_options,
							device_count={"GPU": 1},
							allow_soft_placement=True,
							log_device_placement=False
						)
		#Include to initialise variables. 
		run_metadata = tf.RunMetadata()
		# Run session with specified configs.
		with tf.Session(config=config) as sess:
				saver = tf.train.import_meta_graph(chkpoint+'.meta', clear_devices=True)
				saver.restore(sess, chkpoint)
				opts =tf.profiler.ProfileOptionBuilder.float_operation()
				stats = tf.profiler.profile(sess.graph, run_meta= run_metadata, cmd='op', options=opts)
				opts = tf.profiler.ProfileOptionBuilder.trainable_variables_parameter()    
				params = tf.profiler.profile(sess.graph, run_meta=run_metadata, cmd='op', options=opts)
		return stats,params

\end{lstlisting}



\begin{lstlisting}[language=Python, caption=NVIDIA SMI Logging Tool ,label={lst:logger}]
class GPULogging:
	'''
	GPU statistics logging utility using NVIDIA SMI 
	params: 
	name : str name for ouput log 
	device_id: str ID for GPU device to log
	'''
	def __init__(self, name='',device_id='0'):
		self.name=name
		self.device_id =device_id
		self.log = open(f'/mnt/storage/home/ja17618/{self.name}.out','w+')
		self.process=None
	def start(self):
		self.log.flush()
		if self.process is None:
			self.process = subprocess.Popen(f'nvidia-smi -i {self.device_id} --format=csv -l 1 --query-gpu=index,timestamp,power.draw,utilization.gpu,clocks.current.sm,temperature.gpu,memory.used',stdout = self.log, shell=True)
	def stop(self):
		if self.process is not None:
			self.process.kill()
			self.log.flush()
			self.log.close()
			
\end{lstlisting}
\section{Project Repository}

The full repository containing the software and datasets is contained on a Github repository that can be accessed through the following link: 

\url {https://github.com/jonesmabea/Thesis}


\chapter{Appendix B}
\label{chap:res}
This appendix contains results obtained during hyper-parameter tuning for the VoxelNet network.
\section{VoxelNet Car Class Results}

\begin{table}[H]
	\centering
	\caption{Batch Size-2, Alpha-1, Beta-10, Learning Rate-0.0010, Epochs-160}
	\label{}
	\resizebox{\textwidth}{!}{%
		\begin{tabular}{|l|l|l|l|}
			\hline
			& \textbf{Easy}      & \textbf{Medium}    & \textbf{Hard}      \\ \hline
			2D BB  & 66.694275 & 61.794048 & 56.704201 \\ \hline
			BEV BB & 83.221397 & 71.564476 & 65.815987 \\ \hline
			3D BB  & 44.193409 & 37.166309 & 34.312012 \\ \hline
		\end{tabular}
	}
\end{table}


\begin{table}[H]
	\centering
	\caption{Batch Size-2, Alpha-1.5, Beta-1, Learning Rate-0.001, SGD(Stochastic Gradient Descent), Epochs-160}
	\label{}
	\resizebox{\textwidth}{!}{%
		\begin{tabular}{|l|l|l|l|}
			\hline
			& \textbf{Easy}      & \textbf{Medium}    & \textbf{Hard}      \\ \hline
			2D BB  & 54.1314   & 51.987816 & 49.846    \\ \hline
			BEV BB & 67.7886   & 60.139378 & 57.110561 \\ \hline
			3D BB  & 27.949509 & 24.58765  & 23.919827 \\ \hline
		\end{tabular}%
	}
\end{table}


% Please add the following required packages to your document preamble:
% \usepackage{graphicx}
\begin{table}[H]
	\centering
	\caption{Batch Size-2, Alpha-1.5, Beta-1, Learning Rate-0.001, Epochs-160}
	\label{}
	\resizebox{\textwidth}{!}{%
		\begin{tabular}{|l|l|l|l|}
				\hline
			& \textbf{Easy}      & \textbf{Medium}    & \textbf{Hard}      \\ \hline
			2D BB  & 58.13995  & 55.034439 & 52.658123 \\ \hline
			BEV BB & 71.448044 & 62.753643 & 59.43668  \\ \hline
			3D BB  & 29.088995 & 25.889782 & 25.196236 \\ \hline
		\end{tabular}%
	}
\end{table}


% Please add the following required packages to your document preamble:
% \usepackage{graphicx}
\begin{table}[]
	\centering
	\caption{Batch Size-2, Alpha-1, Beta-10, Learning Rate-0.0010, Epochs-200}
	\label{}
	\resizebox{\textwidth}{!}{%
		\begin{tabular}{|l|l|l|l|}
				\hline
			& \textbf{Easy}      & \textbf{Medium}    & \textbf{Hard}      \\ \hline
			2D BB  & 66.286095 & 60.794079 & 56.602215 \\ \hline 
			BEV BB & 80.853729 & 70.397255 & 65.429283 \\ \hline 
			3D BB  & 40.968906 & 35.093216 & 32.869728 \\ \hline
		\end{tabular}%
	}
\end{table}


\section{VoxelNet Pedestrian Class Results}
% Please add the following required packages to your document preamble:
% \usepackage{graphicx}
\begin{table}[H]
	\centering
	\caption{Batch Size-2, Alpha-1, Beta-10, Learning Rate-0.0010, Epochs-160}
	\label{my-label}
	\resizebox{\textwidth}{!}{%
		\begin{tabular}{|l|l|l|l|}
				\hline
			& \textbf{Easy}      & \textbf{Medium}    & \textbf{Hard}      \\ \hline
			2D BB  & 66.309807 & 60.556213 & 53.479141 \\ \hline 
			BEV BB & 50.308723 & 46.726196 & 41.24551  \\ \hline 
			3D BB  & 34.161884 & 30.576441 & 28.982103 \\ \hline
		\end{tabular}%
	}
\end{table}

% Please add the following required packages to your document preamble:
% \usepackage{graphicx}
\begin{table}[H]
	\centering
	\caption{Batch Size-5, Alpha-1, Beta-10, Learning Rate-0.0010, Epochs-160}
	\label{my-label}
	\resizebox{\textwidth}{!}{%
		\begin{tabular}{|l|l|l|l|}
				\hline
			& \textbf{Easy}      & \textbf{Medium}    & \textbf{Hard}      \\ \hline
			2D BB  & 58.860611 & 57.009262 & 51.562225 \\ \hline 
			BEV BB & 45.534058 & 44.515095 & 40.178875 \\ \hline 
			3D BB  & 27.838013 & 27.808207 & 26.696507 \\ \hline
		\end{tabular}%
	}
\end{table}
