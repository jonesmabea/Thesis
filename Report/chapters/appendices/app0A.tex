%
% file: localoperator.tex
% author: Victor Brena
% description: Briefly describes properties of the local operator.
%

\chapter{Appendix A}


\begin{table}[H]
	\centering
		\resizebox{\textwidth}{!}{%
		\begin{tabular}{|l|l|l|l|l|l|l|}
			\hline
			\textbf{LiDAR}       & \textbf{Hor FOV} & \textbf{Ver FOV} & \textbf{Range} & \textbf{Angular Resolution} & \textbf{Points/second}          & \textbf{Channels} \\ \hline
			\textit{\textbf{VLS-128}} & 360\degree                    & +15\degree to -25\degree          & 300m           & 0.11\degree                               & $\sim$9.6 Million & 128               \\ \hline
			\textit{\textbf{HDL-64E}} & 360\degree                    & 26.9\degree                 & 120m           & $\sim$0.4\degree                          & $\sim$2.2 Million                   & 64                \\ \hline
			\textit{\textbf{HDL-32E}} & 360\degree                    & +10\degree to -30\degree          & 80m-100m       & 0.1\degree                                & $\sim$1.39 Million                  & 32                \\ \hline
			\textit{\textbf{VLP-32C}} & 360\degree                    & +15\degree to -25\degree          & 200m           & 0.1\degree                                & $\sim$1.2 Million                   & 32                \\ \hline
			\textit{\textbf{VLP-16}}  & 360\degree                    & $\pm$ 15\degree                 & 100m           & 0.1\degree                                & 600,000                             & 16                \\ \hline
	\end{tabular}
	}
	
	\caption{Velodyne LiDAR Family}
	\label{velodyne-table}
\end{table}



\begin{lstlisting}[language=bash, caption={.pcap -> .pcd},label={lst:ros}]
	#Code to read .pcap files and convert them to .pcd 
	
	#Load existing pcap file and create a sensor publisher
	$ roslaunch velodyne_pointcloud VLP-16-points.launch pcap:=(path to .pcap file) 
	
	#Receiver to  convert sensor messages to .pcd (In separate terminal window)
	$ rosrun pcl_ros pointcloud_to_pcd input:=/velodyne_points _prefix:=(path to save .pcd file) 
	
\end{lstlisting}


\begin{lstlisting}[language=Python, caption=Calculating FLOPS and Parameters,label={lst:profiler}]
def calculate_metrics(chkpoint):
	"""
	Profiles a Tensorflow Graph to obtain the parameters and FLOPS
	params:
	chkpoint: Tensorflow checkpoint. 
	returns: 
	stats: FLOPS proto object 
	params: Parameters proto objects
	
"""
	with tf.Graph().as_default():
		#GPU options. 
		gpu_options = tf.GPUOptions(per_process_gpu_memory_fraction=0.4,
																visible_device_list='0',
																allow_growth=True)
		#create session config proto object
		config = tf.ConfigProto(
							gpu_options=gpu_options,
							device_count={"GPU": 1},
							allow_soft_placement=True,
							log_device_placement=False
						)
		#Include to initialise variables. 
		run_metadata = tf.RunMetadata()
		# Run session with specified configs.
		with tf.Session(config=config) as sess:
				saver = tf.train.import_meta_graph(chkpoint+'.meta', clear_devices=True)
				saver.restore(sess, chkpoint)
				opts =tf.profiler.ProfileOptionBuilder.float_operation()
				stats = tf.profiler.profile(sess.graph, run_meta= run_metadata, cmd='op', options=opts)
				opts = tf.profiler.ProfileOptionBuilder.trainable_variables_parameter()    
				params = tf.profiler.profile(sess.graph, run_meta=run_metadata, cmd='op', options=opts)
		return stats,params

\end{lstlisting}

\section{Code Listing}

The full code listing is contained on a Github repository that can be accessed through the following link: 

\url {https://github.com/jonesmabea/Thesis/tree/master/src}
