\let\textcircled=\pgftextcircled



\chapter{Literature Review}
\label{chap:lit_review}

In light of the tradeoffs discussed in the previous chapter, different companies working on AVs have come up with different implementations to achieve realisitic AV systems.
It is therefore important to highlight the main modes of operation required in such systems which can be grouped in to three main operations. 
\begin{itemize}
    \item \textbf{Perception} - This is the first step which involves processing the input from the sensors. In this mode tasks such as object detection and tracking, lane detection, traffic sign detection and recognition are performed.
    \item \textbf{Planning} - After the detection and recognition tasks are performed in the perception stage, route and trajectory planning algorithms are performed. These algorithms are required to handle complex situations to ensure safety of the passengers and other road users. 
    \item \textbf{Control} - This stage involves the execution of the plans created in the previous stage. This stage is crucial as the actuators involved in steering and movement have to be able to be able to accurately follow the plans. This involves calculation of energy and forces. At this stage the trajectories and movement of other road users and objects have to be calculated in order to anticipate and avoid any accidents. 
    
\end{itemize}



\subsection{Legal Ethical and Economic Considerations}

The classic ethical dilemma for self driving cars poses a scenario whereby AVs are presented with a situation whereby a fatal accident is inevitable. For example, the AV either has to crash into a group of people in order to save the life of a passenger or to crash itself and sacrifice the life of the passenger. This dilemma highlights important legal, ethical and economic considerations to be considered by companies involved in the production of AVs and their corresponding systems. 


According to \cite{gasser2016fundamental}, there are no public laws that cover the  use of independent autonomous vehicles in public spaces. In his article, he cites the fundamental issue as "understanding and accepting the effect of AVs as an independent action by a machine". Due to the lack of public laws on their use, he proposed the use of extending fundamental rights such as the right to life into the framework for creating laws that cover emerging technologies that have an effect on the public that are otherwise not accounted for in traditional laws. 

Bearing this in mind, it is important to note that despite improvements in traffic safety over the years, 94\% are still caused by human beings with a large majority of them being fatal with the current human error rate being 1 in 100 million miles. This creates a realistic baseline that can then be extended in evaluating the performance of AV systems. As such, if the risk of automation is lesser than the risk of human vehicle control then the AV would be beneficial. Consequently, the AVs are not to be considered as perfect systems.

With regard to economic considerations, car manufucturers involved in the production of AVs have to ensure that their AVs are able to handle numerous scenarios even if they are rare or considered statistically impossible. They should also be required to take legal liability in case any of their system components are defective and result in failures or accidents. This is important for customer trust without which they will not be able to convince customers to buy AVs. In addition, for the adoption of AVs to be widespread, they have to be reasonably priced. With the current prices of the the various components, the adoption of AVs at the moment is not viable. 



A recurring theme that is central to the operation is object detection. Object detection is crucial for safe operation of AVs as it forms the first step before any planning and control. As such, various companies have been working to come up with accurate object detection systems through different combination of sensors in order to achieve this. In addition, to achieve real-time results, the use of Deep Learning techniques have been applied and developed continuously. 


\section{

Following the discussion in the previous section, object detection systems need to fulfill the following requirements. 

\begin{itemize}
    \item \textbf{Robust}
    \item \textbf{Reproducible}
    \item \textbf{Validatable}
    \item \textbf{Viable}
\end{itemize}


These requirements will then form the basis of the evaluating different approaches used to tackle this process. 


\subsubsection{}



Requirements to be fulfilled (Validation/reproducability/Safety/Robustness/ ) \\


\begin{itemize}
    \item \textbf{VoxelNet: End-to-End Learning for Point Cloud Based 3D Object Detection} \cite{zhou2017voxelnet}
    In this paper, they were able to create a point cloud based 3D object detection network by adding a Voxel Feature Extraction layer before a RPN that was able to divide the point cloud into voxels that were used as input for the RPN. 
    This research forms the main basis of my project and will be replicated with the aim of open sourcing it. 
    
     \item \textbf{Complex-YOLO: Real-time 3D Object Detection on Point Clouds} \cite{yolo}
    In this paper, they were able to use point clouds as direct input into Recurrent Neural Networks in order to classify and segment 3D object. They were able to create a global point cloud signature that could be used to achieve this.
    
    \item \textbf{3D Fully Convolutional Network for Vehicle Detection in Point Cloud} \cite{li20163d}
    In this paper, the use of a 3D Fully Convolutional Network(FCN) to detect and localise objects in a point cloud as 3D bounding boxes is discussed. They were able to achieve this by discretising the point clouds into square grids represented by a 4D array containing the dimensions and a channel indicating if there was a point at that space in the grid. This paper will be useful in understanding how the point clouds can be represented in a discretised space.
    
    
    \item \textbf{Faster R-CNN: Towards Real-Time Object Detection with Region Proposal Networks} \cite{ren2015faster}
    In this paper, the idea of real-time RPNs is introduced which combines region-based detectors\cite{girshickICCV15fastrcnn} and CNNs. This is a key paper as the implementation will utilise a RPN that has to be real-time due to the time-sensitive nature of autonomous navigation. 
    
    
    \item \textbf{PointNet: Deep Learning on Point Sets for 3D Classification and Segmentation} \cite{qi2017pointnet}
    In this paper, they were able to use point clouds as direct input into Recurrent Neural Networks in order to classify and segment 3D object. They were able to create a global point cloud signature that could be used to achieve this.
    
    \item \textbf{PointNet++: Deep Hierarchical Feature Learning on Point Sets in a Metric Space} \cite{qi2017pointnet++}
    This paper built up on the PointNet implementation by recursively applying a Hierarchical Neural Network on nested partitions of the point clouds. In doing so they were able to capture the local structures of the point clouds as a result of their metric nature and thus achieve better performance than PointNet. 



\end{itemize}





\subsection{Novelty}



\section{Conclusion}




