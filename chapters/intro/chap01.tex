%
% File: chap01.tex
% Author: Victor F. Brena-Medina
% Description: Introduction chapter where the biology goes.
%
\let\textcircled=\pgftextcircled
\chapter{Introduction}
\label{chap:intro}

In the coming decade there is expected to be proliferation of new technologies that have been aided by recent developments in Artificial Intelligence(AI) and Machine Learning(ML). AI and ML have become topics of increasing interest in academic fields and industries globally. As a result, collaborations between these two fields has become increasingly common. 
With increase in computational power and the development of newer AI algorithms robotic technologies have greatly advanced and are now able to perform complex tasks that were once too difficult, dangerous or even impossible for humans to perform. Consequently, industries have benefited from the accuracy and efficiency provided by these robots. 

An area that has developed a wide range of interest currently is the topic of autonomous vehicles. The idea of autonomous vehicles(AVs) is not new and as early as 2005, DARPA had invested heavily in the creation of unmanned trucks and organised for the Urban Challenge \cite{buehler2009darpa} to allow for different teams to showcase their unmanned vehicles. However, due to the challenges such as low computational power and underdeveloped AI and ML systems, the resulting implementations were not practical and had a high fault rate of 1 fault in 100 miles compared to the human fault rate of around 1 in 100 million miles. Nonetheless, from this challenge, it was clear that the prospect of AVs was plausible and indeed possible. 

Currently there are a number of companies that are developing AVs . These companies have been able to develop vehicles that are capable of certain levels of automation. These levels are defined as:
\begin{itemize}
	\item \textbf{Level 0} - No autonomy. 
	\item \textbf{Level 1} - Basic driver assistance built into vehicle design.
	\item \textbf{Level 2} - Partially autonomous but driver expected to monitor environment at all times.
	\item \textbf{Level 3} - Conditionally autonomous with the driver not required to monitor the environment but is required to take back control if need be.
	\item \textbf{Level 4} - Highly autonomous with the vehicle capable of handling most conditions but the driver has the option to take control. 
	\item \textbf{level 5} - Completely autonomous with the vehicle capable of handling all conditions.
\end{itemize}

\section{Why Autonomous Cars}

\subsection{Safety}
A major argument for self driving cars is the improved safety that they will provide on the road. According to a report released by the United States Department of Transport, National Highway Traffic Safety Administration(NHTSA), around 6,296,000 crashes occured in the year 2015 with 35,092 people losing their lives in these crashes. Of these crashes, around 94\% of them were as a result of human error. Worldwide, it is estimated that there were 1.2 million deaths in 2013 due to road crashes. 
In light of this, self driving cars could greatly reduce the number of road crashes. According to the Eno Centre of Transportation, if 90\% of the cars on the road were autonomous, there would be a reduction of 4,220,000 road crashes, this would save 21,700 lives. This is based on the fact that a large number of road crashes are a as a result of human error and therefore autonomous vehicles will be able to significantly reduce this number. 
However, this estimate is dependent on how well the AV system is designed to be able to handle complex, dynamic driving situations. 

\subsection{Environmental Impact}

According to the Environmental Protection Agency, more than a  quarter of greenhouse gases are from the transportation sector. A major contributor of this is traffic congestion due to various factors such as traffic destabilizing shockwave propagation and road accidents. Autonomous vehicles are able to mitigate this as they are able to gauge and calculate the motion vectors of different objects around them. By using traffic smoothing algorithms and smarter implementations such as the slot mechanism developed by MIT, traffic congestion can be greatly reduced and as a result lower fuel consumption. Furthermore, most companies working with AVs are moving towards electrical vehicles and thus reducing the impact of fossil fuels on the environment. 




\section{Aims and Objectives}
This project aims to evaluate and develop a novel solution to aid perception in AVs by developing an end to end region proposal network capable of accurate object detection from LiDAR point clouds. 

To achieve this aim, the following objectives will be undertaken:
\begin{enumerate}
	\item Detailed analysis of state-of-the-art LiDAR-Based object detection deep neural networks.
	\item Implement voxel feature encoding layer for grouping point clouds
	\item Implement region proposal network for object detection from voxels.
	\item Test and evaluate the implemented neural network against results of state-of-the-art point cloud object detection methods.
\end{enumerate}

\section{Deliverables}

The deliverables are split into technical and analytical. 
\begin{itemize}
	 \item \textbf{End to end point cloud object detection RPN.} This will be a software implementation of the system that will be publicly available through a Github repository.
	\item \textbf{Evaluation report.} In this report, the following topics will be discussed. 
	\begin{enumerate}
		\item A review of related research and implementations tackling object detection using LiDAR cloud points. 
		\item Performance analysis of system and analysis criteria.
		\item A comparison between the implemented system and other state-of-the-art detection systems, potentially through a public benchmark. 
		\item The ethical and safety implications of the system and its viability in a real world setting. 
		\item Economic analysis of the LiDAR-based system and its potential impact on the development of AVs. 
		\item Validation of DNN performance with other public datasets containing data from AVs. 
	\end{enumerate}
\end{itemize}

\section{Added Value}

This project will implement and open source a sophisticated cloud point detection technique for use in autonomous vehicles. The implementation will be able to detect classes of objects such as cyclists, pedestrians and other vehicles on a real-time basis.
In doing so, the project will democratize access to proprietary technology by Apple \cite{zhou2017voxelnet} to be used and developed further by future researchers working with object detection in point clouds. 
Following the detailed evaluation, I intend to propose ways through which this project can be implemented to reduce the cost of AVs in order to make them more viable. 
If successful, this project will provide a possible framework for the main stream adoption of AVs. 

\section{Research scope}
The focus of this project is mainly with regard to computer vision, deep learning and robotics. Computer vision is the task of obtaining, processing, analysing and contextualising visual information to produce numerical information that can be understood and manipulated by computers. This is necessary in order to process and analyse the LiDAR data. Deep learning is a broad term used to describe methods that utilise the use of deep neural networks that have a large number of layers. DNNs have become a heavily researched an invested area due to their ability to capture complex underlying models from data. This will be crucial for detecting objects from point clouds. 
This project will combine existing research using computer vision and deep learning such as \cite{qi2017pointnet}\cite{zhou2017voxelnet} to develop a Region Proposal Network capable of accurately detecting objects in point clouds. 

\section{Report structure}

Chapter 2 discusses the different components of AVs, current implementations in the industry, a background on the research that has been undertaked in the field of object detection and finally a brief overview of deep learning frameworks. 

Chapter 3 is the final discussion that forms the conclusion of the review. In this chapter, I discuss va



