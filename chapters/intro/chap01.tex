%
% File: chap01.tex
% Author: Victor F. Brena-Medina
% Description: Introduction chapter where the biology goes.
%
\let\textcircled=\pgftextcircled
\chapter{Introduction}
\label{chap:intro}

In the coming decade there is expected to be proliferation of new technologies that have been aided by recent developments in Artificial Intelligence(AI) and Machine Learning(ML). AI and ML have become topics of increasing interest in academic fields and industries globally. As a result, collaborations between these two fields has become increasingly common. 
With increase in computational power and the development of newer AI algorithms robotic technologies have greatly advanced and are now able to perform complex tasks that were once too difficult, dangerous or even impossible for humans to perform. Consequently, industries have benefited from the accuracy and efficiency provided by these robots. 

An area that has developed a wide range of interest currently is the topic of autonomous vehicles. The idea of autonomous vehicles is not new and as early as 2005, DARPA had invested heavily in the creation of unmanned trucks and organised for the Urban Challenge to allow for different teams to showcase their unmanned vehicles. However, due to the challenges such as low computational power and underdeveloped AI and ML systems, the resulting implementations were not practical and had a high fault rate of 1 fault in 100 miles compared to the human fault rate of around 1 in 100 million miles. Nonetheless, from this challenge, it was clear that the prospect of autonomous vehicles was plausible and indeed possible. 

\section{Why Autonomous Cars}

\subsection{Safety}
A major argument for self driving cars is the improved safety that they will provide on the road. According to a report released by the United States Department of Transport, National Highway Traffic Safety Administration(NHTSA), around 6,296,000 crashes occured in the year 2015 with 35,092 people losing their lives in these crashes. Of these crashes, around 94\% of them were as a result of human error. Worldwide, it is estimated that there were 1.2 million deaths in 2013 due to road crashes. 
In light of this, self driving cars could greatly reduce the number of road crashes. According to the Eno Centre of Transportation, if 90\% of the cars on the road were autonomous, there would be a reduction of 4,220,000 road crashes, this would save 21,700 lives. This is based on the fact that a large number of road crashes are a as a result of human error and therefore autonomous vehicles will be able to significantly reduce this number. 
However, this estimate is dependent on how well the AV system is designed to be able to handle complex, dynamic driving situations. 

\subsection{Environmental Impact}

According to the Environmental Protection Agency, more than a  quarter of greenhouse gases are from the transportation sector. A major contributor of this is traffic congestion due to various factors such as traffic destabilizing shockwave propagation and road accidents. Autonomous vehicles are able to mitigate this as they are able to gauge and calculate the motion vectors of different objects around them. By using traffic smoothing algorithms and smarter implementations such as the slot mechanism developed by MIT, traffic congestion can be greatly reduced and as a result lower fuel consumption. Furthermore, most companies working with AVs are moving towards electrical vehicles and thus reducing the impact of fossil fuels on the environment. 




\section{History and Current Setups}

In 2009, Google was the first major technology company to announce its self driving car project and within 18 months, it had developed a highly robust system that could handle some difficult roads. In the following years, there was an avalanche of companies that also announced their interest in developing self driving cars ranging from car manufacturers to other technology companies such as Apple, Samsung and NVIDIA.
Startups and other tertiary technology company began investing in developing systems that can be used in these autonomous vehicles. 

Currently, autonomous vehicles are grouped into 5 different categories by the NHTSA:
\begin{itemize}
    \item \textbf{Level 0} - No autonomy. 
    \item \textbf{Level 1} - Basic driver assistance built into vehicle design.
    \item \textbf{Level 2} - Partially autonomous but driver expected to monitor environment at all times.
    \item \textbf{Level 3} - Conditionally autonomous with the driver not required to monitor the environment but is required to take back control if need be.
    \item \textbf{Level 4} - Highly autonomous with the vehicle capable of handling most conditions but the driver has the option to take control. 
    \item \textbf{level 5} - Completely autonomous with the vehicle capable of handling all conditions.
\end{itemize}





\subsection{Componenents of a Self Driving Car}

Most self-driving cars consist of 4 main components: 
\begin{itemize}
    \item \textbf{LiDAR} - LiDAR provides highly detailed 3D information about the evnironment around the vehicle and objects in it. LiDAR operates by sending out pulses of lasers and recording the reflections of the pulses from objects. By comparing this with the time taken for the lasers to be reflected(time of flight) and their direction, the distance of these objects can be calculated and mapped in a point cloud. 
    \begin{equation*}
        distance = \frac{time \times \text{speed of light}}{2}
    \end{equation*}
    
    To achieve a high level of accuracy, the LiDAR has to send out a large enough number of lasers in different directions fast enough to create an accurate point cloud representation of the environment around it. As such, LiDAR systems have multiple channels(emitter/receiver pairs) angled vertically that  emit hundreds of thousands of lasers per second.
    
    \subsubsection{Cost}
    LiDAR systems require complex optical systems that are expensive to build. As such they are the most expensive sensor in AVs. Consequentially, the cost of production of LiDARs increase greatly as the number of channels increase. More channels allow for more accurate representations of the surrounding environment which is necesary for safer navigation of AVs, however this would not be economically feasible. 
    As such, different companies have developed different types of LiDAR in order to still produce accurate point clouds at a reasonable price. 

    \begin{itemize}
        \item \textbf{Mechanical Mirror}
        \item \textbf{Solid State}
        \item \textbf{Optical Phase Array}
        \item \textbf{Microelectromechanical systems (MEMS)}
        \item \textbf{3D Flash}
    \end{itemize}
    
   
    \item \textbf{Cameras} - Cameras mounted on the vehicle are used for classification and identification of various objects on the road. This is important for recognising traffic rules from traffic signs or road markings as well as determining the nature of objects on the road. 
    Cameras can also be used to create 3D maps of the surrounding environment.By combining two cameras, a stereo image can be captured that provides depth information. Alternatively, by combining a camera and IR Laser sensor for depth estimation, RGB-D images are obtained and mapped in a point cloud.

    \subsubsection{Cost}
    
    
    \item \textbf{Position Estimators} - Position estimators are a group of sensors used for navigation of the vehicle. These include GPS systems, odometers and gryometers. 
    \item \textbf{Distance Sensors} - Distance sensors such as radars and sonars are important for gauging the distance of objects on the road. 
    Radars are the most commonly used distance sensors and they work by transmitting radio waves and recording the reflected radio waves from objects. As compared to cameras and LiDARs, radars work well in a variety of low visibility scenarios such as poor weather. 
    However, the reflectivity of these radio waves depends on the nature of objects, their size, absorbtion characteristics and the transmitting power. As such, it is may not be effective for detecting objects with low absorbtion characteristics such as pedestrians and animals.
    
    \item \textbf{Processing Unit} - In order to process all the data from the sensors in the vehicle, AVs require powerful processing units in order to be able to process all this data in real time. Most of the ML/AI algorithms used for detecting and identifying objects from LiDAR and camera data demand large amounts of processing power. This is achieved through the use of CPUs, GPUs, FPGA or combinations with each other. 
    \subsubsection{Cost}
    Due to 
    
    
\end{itemize}


\begin{figure}[h]
    \centering
    \includegraphics[width=\textwidth]{media/waymo.png}
    \caption{Components of Waymo's Self Driving Car (Waymo)}
    \label{fig:my_label}
\end{figure}

As seen from the table above, a wide range of sensors are used in AVs with each . This is essential for accurate navigation of the AV and therefore multiple sensors are fused together in order to provide enough data to achieve this. Given the large number of sensors, a major inhibiting factor in the production of AVs is the cost of sensors. Table \ref{cost} highlights the cost of the various sensors.




- Cost analysis 
- Attempts to minimise cost 
- Tradeoffs 
\section{Conclusion}

As highlighted from the previous section, each of the components serve a crucial purpose in AVs. It is evident that these components complement each other in order to adapt to their shortcomings.

From this a tradeoff between safety, cost, autonomy, power and viability cleary emerges with regard to producing autonomous vehicles. 
In this case, the following terms can be explained as below

\begin{itemize}
    \item \textbf{Safety}
    \item \textbf{Cost}
    \item \textbf{Autonomy}
    \item \textbf{Power}
    \item \textbf{Viability}
\end{itemize}


Cost increase viability decrease, safety increase, autonomy increase. power increase, vice versa is true 
Autonomy increase, cost increase, power increase, safety increase, viability decrease.
Power increase, viability decrease, cost increase










